\usepackage{floatrow}% 浮动体增强宏包
\usepackage{subcaption}% 子图排版
\usepackage{amsmath}% 数学宏包
\usepackage{booktabs}% 三线表格
\usepackage{longtable}% 跨页长表格
\usepackage{multirow,makecell}% 表格行合并,单元格处理
\usepackage{ulem}% 下划线
\usepackage[backend=biber,
                style=gb7714-2015,
                maxbibnames=99,% 著录所有作者
                maxcitenames=2,% 引用标注中最多显示2个作者
                mincitenames=1,% 3个及3个以上的作者截断为1个作者
                gbpub=false,
                gbnamefmt=familyahead,
                url=false,
                doi=false,
                isbn=false,
                gbfieldtype=true, % 输出学位论文标识
                ]{biblatex} % 参考文献

% =========命令行窗口及代码排版(自己开发的宏包)=========
% 请确保工作目录中存在boxie.sty、fvextra.sty和lstlinebgrd.sty三个文件
\usepackage{boxie}
% =========插图标注宏包(修改为标注框可以换行)=========
% 请确保工作目录中存在tikz-imglabels.sty文件
\usepackage{tikz-imglabels}
% =========流程图宏包(自己开发)=========
% 请确保工作目录中存在tikz-flowchart.sty文件
\usepackage{tikz-flowchart}

\usepackage{pgf-umlcd}% UML图宏包
\usepackage[ruled,linesnumbered]{algorithm2e}% 算法排版宏包
\usepackage{siunitx}% 标准单位符号宏包
\usepackage{csquotes}% 引号宏包
\usepackage{hyperref}% hyperref 需要最后引入

%%% Local Variables:
%%% mode: latex
%%% TeX-master:"../main.tex"
%%% End:
